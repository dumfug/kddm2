\documentclass{beamer}
\usepackage{graphicx}
\usepackage{amsmath}


\usetheme{m}
\title{Predicting TCP/IP Network Traffic using Time Series Forecasting}
\subtitle{Interim Presentation}
\date{May 19, 2016}
\author{Thomas Mauerhofer, and Matthias Wölbitsch}


\begin{document}
  \maketitle
  
  \begin{frame}{Recap}   
    \textbf{goal: forecast TCP/IP traffic}
    \begin{itemize}
     \item real-time and short-time
    \end{itemize}
    
    \textbf{data set}
    \begin{itemize}
     \item network traffic of three months
     \item three different resolutions
    \end{itemize}
    
    \textbf{approaches}
    \begin{itemize}
     \item classical time series prediction methods
     \item neural networks
    \end{itemize}
  \end{frame}
 
 
  \begin{frame}{Thomas Approach}
   
  \end{frame}

  
  \begin{frame}{Neural Network Approaches}
    \textbf{neural networks}
    \begin{itemize}
     \item non-linear learning
     \item flexible, powerful
     \item less well behaved
    \end{itemize}
    
    \textbf{feed-forward network}
    \begin{itemize}
     \item multilayer perceptron network
     \item most commonly used for forecasting
     \item sliding window over input series (i.e. set of lags)
     \item one hidden layer with \(n\) neurons
     \item neural network ensemble
    \end{itemize}
  \end{frame}

  
  \begin{frame}{Neural Network Approaches}
    \textbf{recurrent network}
    \begin{itemize}
     \item allows cycles	
     \item long short-term memory (LSTM) architecture
     
    \end{itemize}

    
    \textbf{problems}
    \begin{itemize}
     \item black magic
     \item parameter selection
    \end{itemize}

  \end{frame}

  
  \begin{frame}{Evaluation}
    \textbf{accuracy measures}
    \begin{itemize}
     \item sum squared error (SSE)
     \item symmetric mean absolute percentage error (sMAPE)
     \item \ldots
    \end{itemize}
   
    \textbf{scaled errors}
    \begin{itemize}
     \item compare forecasts on series of different scales 
     \item mean absolute scaled error (MASE)
     \item compare forecast with naïve method 
     \item seasonal version: \(\hat{y}_{t+h,t} = y_{t+h-K}\) %
    \end{itemize}
  \end{frame}

  
  \plain{Questions?}
  
\end{document}