\documentclass{beamer}
\usepackage{graphicx}
\usepackage{amsmath}


\usetheme{m}
\title{Predicting TCP/IP Network Traffic using Time Series Forecasting}
\subtitle{Final Presentation}
\date{June 16, 2016}
\author{Thomas Mauerhofer, and Matthias Wölbitsch}


\begin{document}
  \maketitle
  
  \begin{frame}{Recap}   
    \textbf{goal: forecast TCP/IP traffic}
    \begin{itemize}
     \item real-time and short-time
    \end{itemize}
    
    \textbf{data set}
    \begin{itemize}
     \item network traffic of three months
    \end{itemize}
    
    \textbf{approaches}
    \begin{itemize}
     \item classical time series prediction methods
     \item artificial neural networks
    \end{itemize}
  \end{frame}
  
  \begin{frame}{Holt-Winters Approach}
    \textbf{data set preparation}
    \begin{itemize}
      \item split into training, and test set
      \item important to start at the beginning of a period
    \end{itemize}
    
    \textbf{optimize parameter}
    \begin{itemize}
     \item with grid search
     \item from 0 to 1 with 0.05 steps
    \end{itemize}
    
    \textbf{parameter}
    \begin{itemize}
      \item influence of the previous element
      \item weight for the level ($\alpha$)
      \item weight for the trend ($\beta$)
      \item weight for the seasonality ($\gamma$)
    \end{itemize}
  \end{frame}
  
    \begin{frame}{Results}
    \begin{columns}[c]
    \column{0.35\textwidth}
      \textbf{linear:}
      \begin{itemize}
	\item $\alpha$: 0.1 
	\item $\beta$: 0.2
	\item mase: 1.1768
      \end{itemize}
      
      \textbf{addaptive:}
      \begin{itemize}
	\item $\alpha$: 0.0 
	\item $\beta$: 0.0
	\item $\gamma$: 1.0
	\item period: 7
	\item mase: 0.3168
      \end{itemize}
    
      \column{0.65\textwidth} 
      \textbf{forecast daily traffic}
      \begin{figure}
	\includegraphics[width=1.1\textwidth]{images/daily.png}
      \end{figure}
    \end{columns}
  \end{frame}

  \begin{frame}{Results}
    \begin{columns}[c]
    \column{0.35\textwidth}
      \textbf{linear:}
      \begin{itemize}
	\item $\alpha$: 0.15
	\item $\beta$: 0.1
	\item mase: 4.3413
      \end{itemize}
      
      \textbf{addaptive:}
      \begin{itemize}
	\item $\alpha$: 0.7 
	\item $\beta$: 0.0
	\item $\gamma$: 1.0
	\item period: 24
	\item mase: 3.2444
      \end{itemize}
    
      \column{0.65\textwidth} 
      \textbf{forecast hourly traffic}
      \begin{figure}
	\includegraphics[width=1.1\textwidth]{images/hoursly.png}
      \end{figure}
    \end{columns}
  \end{frame}
  
  \begin{frame}{Results}
    \begin{columns}[c]
    \column{0.35\textwidth}
      \textbf{addaptive:}
      \begin{itemize}
	\item $\alpha$: 0.9 
	\item $\beta$: 0.65
	\item $\gamma$: 0.9
	\item period: 288
	\item mase: 20.8613
      \end{itemize}
    
      \column{0.65\textwidth} 
      \textbf{forecast 5 minute traffic}
      \begin{figure}
	\includegraphics[width=1.1\textwidth]{images/5min.png}
      \end{figure}
    \end{columns}
  \end{frame}

   \begin{frame}{Results}
      \center \textbf{Error-development over the complete test set}
      \begin{figure}
	\includegraphics[width=0.9\textwidth]{images/error.png}
      \end{figure}
  \end{frame}
 
  \begin{frame}{Neural Network Approach}
    \textbf{data set preparation}
    \begin{itemize}
     \item generate sequences using sliding window 
     \item split into training, validation, and test set
    \end{itemize}
  
    \textbf{neural network library}
    \begin{itemize}
      \item keras 
      \item theano
    \end{itemize}
    
    \textbf{hyper parameter search}
    \begin{itemize}
     \item sliding window, number of neurons, number of layers,\ldots
     \item hyperopt library
     \item tree-structured parzen estimator
    \end{itemize}
  \end{frame}
  
  \begin{frame}{Results}
    \begin{columns}[c]
      \column{0.35\textwidth}
      \textbf{MLP}
      \begin{itemize}
	\item \(N=25\)
	\item \(W = \{1,2,4,8\} \cup \{287,288,289\}\)
      \end{itemize}
    
      \textbf{1 layer LSTM}
      \begin{itemize}
	\item \(N = 19\)
	\item \(W = \{1,2,\ldots,19\}\)
      \end{itemize}
      
      \textbf{2 layer LSTM}
      \begin{itemize}
	\item \(N_1 = 13\)
	\item \(N_2 = 5\)
	\item \(W = \{1,2,\ldots,14\}\)
      \end{itemize}
    
      \column{0.65\textwidth} 
      \textbf{forecast error for different horizons}
      \begin{figure}
	\includegraphics[width=1.1\textwidth]{images/nn_errors.pdf}
      \end{figure}
    \end{columns}
  \end{frame}
  
  \begin{frame}{Results}
    \textbf{forecasting examples with \(h=1\) and \(h=24\) using MLP}
    
    \begin{columns}[c]
      \column{0.5\textwidth}  
      \begin{figure}
       \includegraphics[width=1.1\textwidth]{images/example_forecast.pdf}
      \end{figure}

      \column{0.5\textwidth}
       \begin{figure}
        \includegraphics[width=1.1\textwidth]{images/example_forecast_day.pdf}
       \end{figure}
    \end{columns}
  \end{frame}
  
  
  \begin{frame}{Conclusion}  
    \begin{columns}[c]
      \column{0.5\textwidth}  
      \textbf{forecast horizon}
      \begin{itemize}
       \item one step ahead forecasting
       \item direct vs. iterative forecasting
      \end{itemize}
      
      \hspace{10pt}
      
      \textbf{training loss function}
      \begin{itemize}
       \item MSLE 
       \item penalizes underestimates
       \item numerical issues
      \end{itemize}

      \column{0.5\textwidth}
        \begin{figure}
         \includegraphics[width=1.0\textwidth]{images/msle_standardized_data_issue.png}
        \end{figure}
    \end{columns}
  \end{frame}
  
  
  \begin{frame}{Conclusion}
    \textbf{LSTM issues}
    \begin{itemize}
     \item high expectations
     \item too few training samples
     \item slow
    \end{itemize}
    
    \textbf{neural networks and time series}
    \begin{itemize}
     \item used often for forecasting
     \item numerous different approaches
     \item problem solved?
    \end{itemize}
  \end{frame}

  \begin{frame}{Comparison}
   \textbf{two very different methods}
   \begin{itemize}
    \item neural networks: black box
    \item Holt-Winters: time series engineering 
   \end{itemize}
   
   \hspace{10pt}
   
   \textbf{→ hard to compare}
  \end{frame}

  
  \plain{Questions?}
\end{document}